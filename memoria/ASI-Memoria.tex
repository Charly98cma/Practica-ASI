\documentclass[12pt,a4paper, spanish]{article}
\usepackage[spanish]{babel}
\usepackage{setspace}
\usepackage{float}

% Enter the name/s of the author/s, the document title and his subject
\usepackage[
  pdftex,
  pdfauthor={Carlos Miguel, Antonio Ramírez, Younes Aouad},
  pdftitle={Memoria Práctica ASI - Grupo 30},
  pdfsubject={Memoria práctica ASI}
]{hyperref}

% --- TIKZ ---
\usepackage{tikz}
\usepackage[edges]{forest}
% --- TIKZ ---


% --- IMAGES ---
\usepackage[pdftex]{graphicx}
\usepackage{subfig}
\usepackage{graphicx}
\usepackage[usenames,dvipsnames]{color}
\DeclareGraphicsExtensions{.png,.jpg,.pdf,.mps,.gif,.bmp}
% --- IMAGES ---

% --- MARGIN DIMENSIONS ---
\frenchspacing \addtolength{\hoffset}{-1.5cm}
\addtolength{\textwidth}{3cm} \addtolength{\voffset}{-2.5cm}
\addtolength{\textheight}{4cm}
\setlength{\headheight}{15pt}
% --- MARGIN DIMENSIONS ---

% --- HEADLINE ---
\usepackage{fancyhdr}
% Name of the subject this work is for
\fancyhead[R]{Administración de Sistemas Informáticos}\fancyhead[L]{\today}
\fancyfoot[C]{\rule{1cm}{0.5mm}\\\thepage} \pagestyle{fancy}
% --- HEADLINE ---



% --- DOCUMENT ---
\begin{document}

% --- TITLE PAGE ---
\begin{titlepage}
  \newcommand{\HRule}{\rule{\linewidth}{0.5mm}}
  \centering

  % --- SECONDARY TITLES ---
  \textsc{}\\[0.25cm]

  % Name of your university
  \textsc{\huge{Universidad Politécnica de Madrid}}\\[0.5cm]

  % Name of your faculty
  \textsc{\LARGE Escuela Técnica Superior\\de Ingeniería Informática}\\[0.3cm]

  \begin{figure}[H]
    % Replace this with your university logo
    \centering
    \subfloat{{\includegraphics[width=7cm]{Images/LogoUPM}}}
    % Replace this with you faculty logo
    \qquad
    \subfloat{{\includegraphics[width=2.75cm]{Images/LogoFI}}}\\[0.5cm]
  \end{figure}

  % Name of the department to which the subject belongs.
  \textsc{\Large Departamento de Lenguajes, Sistemas Informáticos e Ingeniería de Software}\\[0.25cm]

  % Name of the subject
  \textsc{\large Administración de Sistemas Informáticos}\\[0.25cm]
  % --- SECONDARY TITLES ---

  % --- MAIN TITLE ---
  \HRule\\[0.4cm]

  % Title of this work
  {\huge\textbf{Memoria de Proyecto Práctico}\\[0.4cm] \textit{Script maestro para la configuración\\de un cluster Linux}}\\[0.4cm]
  \HRule\\[1.25cm]

  % --- SEMESTER and SCHOLAR YEAR ---
  \textsc{\large 4º Curso | 7º Semestre}\\[1cm]
  % --- SEMESTER and SCHOLAR YEAR

  % --- AUTHORS ---
  {\large\underline{\textit{Autores}}}\\[0.2cm]
  \textsc{Ramírez Solans, Antonio}\\
  Nº Matrícula: 170035\\[0.2cm]
  \textsc{Miguel Alonso, Carlos}\\
  Nº Matrícula: 170243\\[0.2cm]
  \textsc{Aouad Idrissi Boulid, Younes}\\
  Nº Matrícula: 170155\\[0.2cm]
  % --- AUTHORS ---

  % --- DATE ---
  \vfill\vfill\vfill
      {\large\today}
      % --- DATE ---

      % --- MAIN TITLE ---

\end{titlepage}
% --- TITLE PAGE ---


\newpage
% --- INDEX ---
\thispagestyle{empty}
\pagenumbering{gobble}
\renewcommand*\contentsname{Índice de Contenidos}
\tableofcontents
% --- INDEX ---

\newpage

\pagenumbering{arabic}
\section{Introducción del proyecto}
Este proyecto se basa en el desarrollo de un \textit{script} (o conjunto de ellos) en Bash que permitan la configuración de diferentes servicios en un \textit{cluster} formado por múltiples máquinas ejecutando un sistema Linux.\\

De esta forma, el administrador puede configurar diferentes servicios en diferentes máquinas de forma simultánea, teniendo que escribir solo los archivos de configuración de cada servicio y el archivo de configuración general.\\

\noindent Los servicios que soportados por el \textit{script} de configuración son:
\begin{itemize}
\item Mount
\item Raid
\item LVM
\item NIS (cliente + servidor)
\item NFS (cliente + servidor)
\item Backup (cliente + servidor)
\end{itemize}

\newpage

\section{Desarrollo del proyecto}
\subsection{Decisiones de diseño}

\noindent Durante el desarrollo de la práctica, vimos que había ciertos aspectos de ella que podían realizarse de diferentes maneras, tales como la ejecución de los comandos en la máquina destino, el formato del/los \textit{script/s}, etc\ldots\\

\noindent Por ello, se han tomado la siguientes decisiones:
\begin{itemize}
\item Tener un \textit{script} central que ejecutará el administrador (\texttt{configurar\_cluster.sh}) que leerá línea a línea el fichero de configuración que se proporciona como argumento, y llamará a la función perteneciente al servicio leído, que está localizada en la carpeta \texttt{lib}, cada servicio en un fichero Bash diferente. Además, ciertas funciones se han puesto en común en el archivo \texttt{aux\_functions.sh}, de forma que todos los ficheros que lo requieran puedan acceder a estas funciones de forma sencilla y sin tener código repetido multitud de veces.

  A continuación se muestra el árbol de directorios de los \textit{scripts} del proyecto (no incluido directorio de tests, que se verá en la sección \textsl{Testing}):\\
  \begin{center}
    \begin{forest}
      for tree={%
        folder,
        grow'=0,
        fit=band,
      }
      [/scripts
        [configurar\_scripts.sh]
        [lib
          [aux\_functions.sh]
          [serv\_mount.sh]
          [serv\_raid.sh]
          [serv\_lvm]
          [serv\_nisS.sh]
          [serv\_nisC.sh]
          [serv\_nfsS.sh]
          [serv\_nfsC.sh]
          [serv\_backupS.sh]
          [serv\_backupC.sh]
        ]
      ]
    \end{forest}
  \end{center}

\item Para facilitar la ejecución e implementación, se ha decidido que los comandos se ejecutarán a través de \texttt{SSH}, en vez de enviar un fichero de configuración y ejecutarlo. Esta decisión aumenta el tráfico de la red, ya que se tienen que enviar los comandos uno a uno, tanto para comprobaciones como para configuraciones, pero permite un control más preciso de la ejecución.
\end{itemize}


\section{Servicios implementados}
\noindent A continuación se explicarán los diferentes servicio que se han implementado y su funcionamiento general, incluyendo cualquier detalle relevante a la hora de entender su funcionamiento.\\

Las funciones de cada servicio reciben 4 argumentos, que son, el fichero de configuración (\texttt{fichero\_configuración}), la dirección del host (ya sea una dirección IP o un nombre), el fichero de configuración del servicio (e.g. \texttt{raid.conf}) y el número de línea del fichero de configuración en el que se han leído los datos anteriores.\\

De igual forma, en todas las funciones se ha re-direccionado la lectura del fichero de configuración de cada servicio por el \textbf{descriptor de fichero 3}.\\

Todos los \textit{scripts} de los servicios retornarán 0 si se ha realizado satisfactoriamente la operación, y, si se ha producido algún error, un valor específico representando ese error (ver \textit{Códigos de Error}).


\subsection{Mount}
\noindent El servicio \textit{mount} leerá de su fichero de configuración dos campos, el dispositivo a montar y el punto en el que montar dicho dispositivo.\\

El \textit{script} montará el dispositivo en el punto de montaje tras comprobar que se han leído todos los campos necesarios, que dicho dispositivo existe en la máquina, y que el punto de montaje es un directorio vacío. Si el punto de montaje no existe, se crea.\\


\subsection{Raid}
\noindent El servicio \textit{raid} leerá 3 campos de su fichero de configuración, el nombre del nuevo dispositivo RAID, el nivel de RAID, y los dispositivos que compondrán el RAID.\\

Lo primero que hace el \textit{script} es comprobar si la herramienta \textit{mdadm}, que usará para la instalación del RAID, está instalada, si no, la instalará.\\

Tras esto, comprueba la correcta lectura de los campos, tras lo que se comprueba que el nivel de RAID leído es uno de los niveles soportados (0, 1, 5, 6 o 10). Tras esto, se comprueba que cada dispositivo leído que compondrá el RAID no tiene un sistema de ficheros previo y, si todos estos requisitos se cumplen, el \textit{script} creará de forma correcta el RAID.\\


\newpage
\subsection{LVM}
\noindent El servicio \textit{lvm} lee las dos primeras líneas, que contienen el nombre del grupo del RAID y los volúmenes físicos que lo compondrán. Tras esto, leerá, línea a línea, cada uno de los volúmenes lógicos a crear en el RAID.\\

Primero comprueba que la herramienta a usar, en este caso, \textit{lvm2}, y todos sus componentes, estén instalados. Tras esto, verifica la correcta lectura de las dos primeras líneas, comprueba que cada volumen físico exista, crea el grupo y añade los volúmenes a él.\\

Si no han surgido errores, el \textit{script} leerá cada uno de los volúmenes lógicos, comprobando que no se exceda el tamaño máximo ni la capacidad de almacenamiento.


\subsection{NIS}
\subsubsection{Servidor}
\subsubsection{Cliente}

\subsection{NFS}
\subsubsection{Servidor}
\subsubsection{Cliente}

\subsection{Backup}
\subsubsection{Servidor}
\subsubsection{Cliente}


\newpage
\section{Códigos de error}
Para facilitar el desarrollo y la implementación de todas las comprobaciones, errores y mensajes, se ha asignado a cada servicio un rango de 10 posibles códigos de error, los cuales están especificados a continuación:

\subsection{Comúnes a todos los servicios}
Rango de códigos de error: 1 - 9 y 255
\begin{itemize}
\item \textbf{1}: No se ha proporcionado fichero de configuración a \texttt{configurar\_cluster.sh}
\item \textbf{2}: El fichero de configuración no existe o es un directorio
\item \textbf{3}: Error en el formato del fichero de configuración
\item \textbf{4}: Un fichero de configuración de un servicio no existe
\item \textbf{5}: Servicio desconocido en el fichero de configuración
\item \textbf{6}: Error en el formato del fichero de configuración de un servicio
\item \textbf{255}: Error del servicio SSH
\end{itemize}

\subsection{Mount}
Rango de códigos de error: 10 - 19
\begin{itemize}
\item \textbf{10}: El dispositivo a montar no existe
\item \textbf{11}: El punto de montaje no es un directorio vacío
\item \textbf{12}: Error inesperado durante el montaje
\item \textbf{13}: Error inesperado al crear el directorio
\end{itemize}

\subsection{Raid}
Rango de códigos de error: 20 - 29
\begin{itemize}
\item \textbf{20}: Error inesperado al configurar el RAID
\item \textbf{21}: El nivel de RAID no es válido (no soportado)
\end{itemize}

\subsection{LVM}
Rango de códigos de error: 30 - 39.
\begin{itemize}
\item \textbf{30}: Uno de los dispositivos especificados no existe
\item \textbf{31}: Error inesperado al inicializar los volúmenes físicos
\item \textbf{32}: Error inesperado al crear el grupo de volúmenes físicos
\item \textbf{33}: Se ha excedido el tamaño del grupo al crear los volúmenes lógicos
\item \textbf{34}: Error inesperado al crear uno de los volúmenes lógicos
\end{itemize}

\subsection{NIS}

\subsubsection{Servidor}
Rango de códigos de error: 40 - 49.
\subsubsection{Cliente}
Rango de códigos de error: 50 - 59.

\subsection{NFS}

\subsubsection{Servidor}
Rango de códigos de error: 60 - 69.

\subsubsection{Cliente}
Rango de códigos de error: 70 - 79.

\subsection{Backup}

\subsubsection{Servidor}
Rango de códigos de error: 80 - 89.

\subsubsection{Cliente}
Rango de códigos de error: 90 - 99.

\newpage
\section{Testing}

\end{document}
% --- DOCUMENT ---
